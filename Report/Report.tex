\documentclass{article}

\usepackage{multirow}
\usepackage{verbatim}
\usepackage{graphicx}
\usepackage{fullpage}
\usepackage{amsmath}
\usepackage{amssymb}
\usepackage{cancel}
\usepackage{hyperref}
\usepackage{ulem}
\usepackage[usenames,dvipsnames]{color}
\newtheorem{claim}{Claim}




% allows for temporary adjustment of side margins
\usepackage{chngpage}

% just makes the table prettier (see \toprule, \bottomrule, etc. commands below)
\usepackage{booktabs}


\newcommand{\pd}[2]{\frac{\partial #1}{\partial #2}}
\begin{document}

\title{Project 1: A Python Interpreter for the Browser}
\author{Ian Gemp and Shaylyn Adams}
\maketitle

\section{Introduction}
The goal of this project is to write a typescript program that could interpret python bytecode (.pyc) in the browser.  Our general approach was to split this task into three parts: parsing the bytecode, interpreting the result, and porting to the browser.  All preliminary work was to be done outside the browser using node.js with the last step being the transition to the browser using browserFS.\\
\\
When python (.py) files are executed, they are compiled into bytecode (.pyc) and interpreted by the python interpreter (e.g. CPython, PyPy, etc.).  The bytecode itself is conveniently universal across the interpreters of the same version (e.g. Python 2.7).  In all subsequent executions of the python file (.py), the interpreter will run the bytecode (.pyc) directly in order to save time re-compiling the python file (.py) assuming the python file (.py) has not changed since its last compilation.  In reality, the bytecode (.pyc) must be read and parsed in its entirety before the python implementation actually executes the source.

\section{Parsing}

The python bytecode (.pyc) adheres to a strict format, the ``Marshall'' format, that can be read and interpreted by the python implementation (e.g. CPython).  The majority of the bytecode is structured in a way such that the interpreter can read in the bytecode stream as an alternating sequence of datum (singular piece of data) and datum types.  Some of the datum types represent containers (e.g. objects, namespaces, etc.) and a natural hierarchy of data results.

\subsection{Parsing Dependencies}

Our program assumes the python bytecode was compiled with Python 2.7 although we believe the marshal format is consistent across the 2.x line.

\subsection{Understanding Python Bytecode}

As mentioned above, the majority of the python bytecode is structured as an alternating series of types and data chunks, however the first 8 bytes (64 bits) of bytecode are specially formatted.  The first 4 bytes (32 bits) of bytecode represent a magic number corresponding to the marshal format version.  The next 4 bytes are the modification timestamp of the python source file (.py).  The next byte constitutes the beginning of the type/datum sequence.  Each ``type-byte'' is stored in the bytecode as an unsigned 8 bit integer representing a single character.  This character corresponds to a specific datum type, which dictates how the following bytes should be read in order to correctly parse the corresponding datum.  For instance, the type-byte ``s'' indicates that the datum is a string.  The 4 bytes (32 bits) following the type-byte are a signed integer representing the size of the string to be read.  Once the string is parsed, the program moves to the following byte and expects to encounter another type-byte.  The only way this rule is broken is if the program has moved beyond the length of the bytecode file, in which case parsing is complete.  The diagram below depicts this example and the general algorithmic scheme for parsing the bytecode.\\
\\
0) Initialize Global Interned List \\
1) Read Magic Number \\
2) Read Timestamp \\
3) While Not End of File \\
4) \indent Parse\_Type\_Datum() \\

\subsection{Special Cases}

64 Bit, Complex Numbers, Floats, interned\_list?

\section{Interpreter}

\subsection{Interpreter Dependencies}

\subsection{Understanding the Interpreter}

\subsection{Special Cases}

\section{Conclusion}




















\end{document}